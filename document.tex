%%This is a very basic article template.
%%There is just one section and two subsections.
\documentclass[a4paper,12pt]{article}
\usepackage[T2A]{fontenc} 
% \usepackage[cp1251]{inputenc}
\usepackage[utf8x]{inputenc}
\usepackage[english,russian]{babel}
\usepackage{amssymb,amsfonts,amsmath,mathtext,cite,enumerate,float}
\usepackage{hyperref}
\usepackage{cite}

%\usepackage{xcolor}
% \newcommand\myworries[1]{\textcolor{red}{#1}}


\title{Современная экономическая система России} % Заглавие документа
\author{Федор Вомпе, 507 группа}
\date{\today} % Дата создания


\begin{document}

\maketitle

\section{Введение}

Несмотря на рецессию в 2009 году, среднегодовой темп роста экономики России за
последние 12 лет составил более 5\% в год\cite{TheEconomist}. По данным
Организации экономического сотрудничества и развития 
\footnote{ \textbf{Организация экономиического сотруудничества
и развиития} (сокр. ОЭСР, англ. Organization for Economic Co-operation and
Development, OECD) — международная экономическая организация развитых стран,
признающих принципы представительной демократии и свободной рыночной экономики.
- \textit{источник
\href{http://ru.wikipedia.org/wiki/Организация экономиического сотруудничества
и развиития}{Википедия}} } темпы роста экономики России замедлятся 
до 4\% в этом и следующем году\cite{StandardPoor}. В то время как инфляция может
вырасти по сравнению 6,1\% в прошлом году\footnote{По данным Росстата и журнала The
Economist}. Уровень безработицы в настоящее время ниже среднего по данным ОЭСР.

В отличие от стойкого дефицита бюджета в 1990 году, за последние пару лет
в России наблюдается стойкий профицит бюджета. Все это благодаря
высокой и растущей цене за нефть, экономический росту, а также
благодаря налоговой реформе.

Целью этого эссе является исследование современной экономической
системы России, а также выявление основных экономических факторов, влияющих на
развитие экономики России.  

\section{Структура экономики России}

Рост реального ВВП России в 2011 году составил 4\% и может составить эту же
цифру в связи c усилением рисков замедления роста мировой экономики,
вызванных замедлением роста экономик США и Евросоюза, а также продолжающегося 
долгового кризиса в Европе и ожидаемого снижения цен на
нефть. \cite{WorldBank2011}

В этой части эссе будут рассмотрены основные составляющие экономики России, в
особенности все 

\TODO{технологический уклад}

\subsection{Добывающая промышленность и природные ресурсы}


\section{Заключение}
Доходы от нефти и газа, которые в 2008 году составили треть всех государственных
доходов (около 200 миллиардов долларов), были использованы для погашения 
внешнего долга и для наращивания активов в стабилизационный фонд, который
недавно был использован для введения финансовых стимулов. Бюджет России все
еще сильно зависит от цен на нефть. Если бы не было нефти, то еще с 2005 года
экономический рост России серьезно замедлился.

Сокращение зависимости государственного бюджета от доходов на нефти
позволит укрепить и модернизировать экономику России.


\bibliography{myrefs}{}   % expects file "myrefs.bib"
\bibliographystyle{plain} % (uses file "plain.bst")

\end{document}
